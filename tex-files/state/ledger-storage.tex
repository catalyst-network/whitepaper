
The global ledger state is stored locally by each node in a key value pair database. Any performant database such as RockDb can be used to store the state data, however the data must be organised using a structure called a Sparse Merkle Tree. This structure allows the use of a single hash called the root hash to represent the entire data state. Any changes to the data would result in a completely different root hash, therefore it can be used by nodes to verify that their dataset is the same as that of other nodes. Consequently, a node using an alternative storage organisation would be unable to interact with the network due to an inability to calculate the same root hash.

The Sparse Merkle Tree data structure allows for:

\begin{enumerate}
\item Persistent storage of the current and previous states of the ledger.
\item Efficient provision of proof that queried data exists, or importantly, does not exist in the ledger.\end{enumerate}

In addition to storing the global ledger state in a Sparse Merkle Tree, the smart contract KVM code and smart contract data are also stored as a Sparse Merkle Tree. This allows the root hash of both these data sets to be stored in the \emph{storageRoot} and \emph{codeHash} fields in the account state.

\subsubsection{Merkle Tree}

A Merkle tree is a structure which allows a large set of data to be committed to with only a short string known as the root hash. If any of the data is changed, the root hash will also change. This allows honest nodes to recognise and reject information that has been altered outside of the ledger cycle by malicious parties. \\

Each item of data being stored (here the serialised account information) is represented as a leaf of the Merkle tree. The data in neighbouring leaves is hashed and concatenated and this value is used for the node one level above in the tree. Each node is stored in the database with its lookup key being the hash of its value. every leaf node is labelled with the hash of a data block, and every non-leaf node is labelled with the cryptographic hash of the labels of its child nodes. The nodes are themselves combined until a single hash has been obtained. This is the root hash and its value depends on every bit of data stored in the tree.\\

\begin{figure}[h]
  \centering
\includegraphics[scale=0.8]{merkle-tree}
\end{figure}

To store or retrieve an entry from a Merkle trie, a key is used. For the account storage we use the account address as the key. This key is not equivalent to a key in the underlying database, instead it is used to determine the path from the root hash to the data entry. The key tells us which branch to follow as we progress down the tree. \\

When a data entry changes, every node above it in the tree must be recalculated. The hash of the data held by a node is used as its key in the database, therefore every recalculated node will constitute a new entry in the underlying database, rather than an existing entry that needs to be updated.\\

\subsubsection{Merkle Proof}

A client with knowledge of the current root hash of the state tree does not need to retrieve the complete dataset to verify the inclusion of a specific entry. They only need the information required to reconstruct the root hash with the specific entry. This requires knowledge of each sibling hash along the path down to the entry being authenticated. For example, in the diagram above, the inclusion of entry $a$ can be verified with just $a$, $H(A)$, and $H(c)+H(d)$. If the newly calculated root hash is not 

However how do we prove non-inclusion?

\subsubsection{Sparse Merkle Tree}

The Sparse Merkle Tree 


\subsubsection{Choice of Sparse Merkle Tree over Merkle Patricia Tree}

-balanced
-easy proof of non existence
-simpler to implement

