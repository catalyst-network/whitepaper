The InterPlanetary File System forms the basis of the Catalyst DFS. IPFS is simply a file system, this can be used to store retrieve and distribute files to other peers. It is itself based on a distributed network, meaning that there is no one central entity that holds and controls the flow of information. Across the IPFS network nodes hold as little or as much data as the wish with multiple copies of the data being held across many unrelated nodes. This means that there is no single point of failure in the network. An attempted attack on the IPFS would require a majority of nodes to be taken offline to prevent access to the files held on the network.   \\

IPFS uses a bittorrent style \\

An advantage that IPFS has compared to traditional computer filing systems where the URL or file path describe its location, on IPFS what is contained in the file is declared, this is called content addressing, this is using a Content Identifier (CID). This CID is a hash of the contents of the file, thereby changing the contents would change the hash and thereby the CID. Through content addressing, a malicious entity would not be able to send a user a false file in the place of the on that they have requested. This is due to the hardness of reversing a hashing function. The CID also means that all files stored on IPFS are immutable i.e. can not be edited as changing anything contained within the file means that that CID for that file will now be changed. Through linkage of these files it means that a version history of files can be created. \\

Once a file has a CID associated with it a distributed hash table can be formed. Distributed hash tables are key-value stores, which which the elements are held as hashes. The CID becomes the key element. Any peers that hold the file have their ID's associated with the file. The CID associated with a file means that the file is not duplicated on the network. For example if Alice uploads file $x$ it is stored on IPFS, other users can then download and hold the file, if then Bob uploads the same file $x$ then because the CID for both files would be the same the duplication would not take place as the file is already held on the file system.  \\

IPFS integrates Merkle Directed Acyclic Graphs (Merkle DAGs). Merkle DAGs are used to form linkages between blocks of information. For example if an entire folder is added to IPFS, the entire folder would have a CID, but additionally the constituent elements would also. The leaf nodes would be formed of the constituent files. Merkle DAG's are used compared to traditional Merkle trees as the allow a flow or direction of information to be established. Thereby allowing an order for the reading of the files to be set. 
