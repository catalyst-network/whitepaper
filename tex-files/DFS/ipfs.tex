The InterPlanetary File System forms the basis of the Catalyst DFS. IPFS is a collection of protocols which specify how nodes are supposed to operate within the network, and various multilingual implementations of these protocols. As it is distributed, there is no one central entity that holds and controls the flow of information and most importantly who has access. Across the IPFS network nodes hold as little or as much data as the wish with multiple copies of the data being held across many unrelated nodes. This means that there is no single point of failure in the network. An attempted attack on the IPFS would require a majority of nodes to be taken offline to prevent access to the files held on the network.   \\

IPFS uses a bittorrent style seeding system. Meaning that the holder of any particular file that is online will seed the file to other peers that request the file. This means that the more peers on the network the more efficiently files can be distributed. IPFS user have the ability to `pin' files meaning that those files are stored and secured by the peer. IPFS works in such a way that the more popular \textit{i.e.} the more peers that have pinned the file, the more easily accessible that file will be.   \\

In traditional computer file systems, files are indexed, referenced, and accessed through their location: for instance, \verb`~/Documents/file.md` would refer to a Markdown file inside a "Documents" folder inside the home directory of a user on UNIX systems. However, on a distributed system, there can be no global state that could be used for addressing content. Instead of location-addressing, IPFS and other distributed systems use content-addressing. The CID also means that all files stored on IPFS are immutable i.e. can not be edited as changing anything contained within the file means that that CID for that file will now be changed. Through linkage of these files it means that a version history of files can be created. \\

In a content-addressed system, every file on the system is given a unique, deterministic content identifier (CID) by hashing the content. Hashing works by deterministically using the contents of the file as the input for a standard encryption algorithm. Changing the contents of a file would change the hash of the file. This CID can then be used to index, reference, and access the file directly, regardless of where it is. Advantageously, through content addressing, a malicious entity would not be able to send a user a false file in the place of the one that they have requested. This is due to the difficulty of reversing a hashing function. The CID associated with a file means that the file is not duplicated on the network. For example if Alice uploads file $x$ it is stored on IPFS, other users can then download and hold the file, if then Bob uploads the same file $x$ then because the CID for both files would be the same the duplication would not take place as the file is already held on the file system.  \\

Once a file has a CID associated with it, a distributed hash table (DHT) can be formed. DHTs are key-value stores, where the key is the CID of the relevant file, and the value is an array containing all of the user IDs of any peer which holds the file. The DHT is used by peers looking for a file to find the peers that it can retrieve it from. Once a user has retrieved the file from another peer in the DHT, it can also then be added as a peer that holds the file. \\

IPFS integrates Merkle Directed Acyclic Graphs (Merkle DAGs). Merkle DAGs are abstract trees which result when multiple CIDs are used as the input for another hashing algorithm, resulting in chains where entire hierarchies of files can be referenced using a single CID. Any change to any constituent file at the bottom of the tree (the leaf nodes) percolates up to the next hash, and the next, causing the CID for the entire tree to change. Merkle DAG's are used compared to traditional Merkle trees as the allow a flow or direction of information to be established. Thereby allowing an order for the reading of the files to be set. 
