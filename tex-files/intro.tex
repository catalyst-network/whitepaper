With the creation of bitcoin, we witnessed the opening of pandora's box in terms of financial ingenuity, self-sovereignty and a shift of power from the financial elite to the people. Since the 80's people like David Chaum theorised the ideas of distributed consensus-based protocols. Although a technical and social breakthrough bitcoin now comparatively is like sending money via cheques in terms of the raw throughput and functionality.
\\\\
Again with the creation of Ethereum we witnessed the opening of pandora's box for a second time. We were introduced to the concept of the world state and distributed computing which until its creation had only been theorized by people like Nick Szabo 20years ago. The idea of decentralised finance, peer-based crowdfunding further challenged the grasp of the financial world. But at this time new societal concepts such as web3 were introduced.
\\\\
The web3 movement has become a war cry for a new and better web. With the rise in popularity in consensus protocols, where users demand more control over the finances, we are now demanding more privacy, more security and more control over our data. Web3 is the idea that the same stateful protocols that are revolutionising the financial industry can also revolutionise the much-broken web.
\\\\
Consensus-based protocols like blockchains introduced a method for each participant in a network to hold and transfer state in a digitally native format, applying stateful protocols to the traditionally stateless protocols that power the web, gives the web a native mechanism to authoritatively say who owns what, who has permissions to do an action.
\\\\
But again as groundbreaking a technology Etheruem is, and the improvements it brings in interns of technical functionality as a distributed system and highlighting benefits it can bring to all parts of our life. Ethereum as one component of the web3 ecosystem has some major problems in terms of scalability feature-rich functionality. More generally with the current web3 paradigm trying to build decentralised applications often forces developers to adopt multiple protocols, multiple native tokens, and multiple technology stacks in return to offer simplistic functionality compared to our stateless ancestor protocols.
\\\\
The Catalyst team looked at the "state" of Web3 it's self and said, it's good, but it can still be better. 
Our approach to building Catalyst was to solve some core issues

\begin{itemize}
\item Build a true Byzantine fault-tolerant consensus that becomes increasing scalable, decentralised and secure at scale  .
\item Provide a feature-rich full-stack solution to developers who want to embrace the web3 ethos.
\item Rethink the economic incentives around blockchain systems, to be fairer to all participients.
\end{itemize}


\begin{comment}
NB - To be moved to the DFS section?

Catalyst solves this problem through the integration of a Distributed File System (DFS). This enables much more control for peers on the network in terms of what elements of the ledger they hold. This means that lightweight nodes can run with only a subset of the network data. Furthermore, large data files can be securely held on the ledger without causing issues for other nodes due to bloating.


\end{comment}