With the creation of Bitcoin \cite{nakamoto2008bitcoin}, we witnessed the opening of Pandora's box in terms of financial ingenuity, self-sovereignty and a shift of power from the financial elite to the people. Since the 1980's people like David Chaum have theorised about the ideas of distributed consensus-based protocols \cite{chaum1979computer}. Although a technical and social breakthrough, Bitcoin now comparatively is like sending money via cheques in terms of the raw throughput and functionality.
\\\\
With the creation of Ethereum \cite{wood2014ethereum}, we witnessed the opening of Pandora's box for a second time. We were introduced to the concept of the world state and distributed computing which until its creation had only been theorized by people like Nick Szabo 20 years ago. The idea of decentralised finance and peer-based crowdfunding further challenged the grasp of the financial world. At this time new societal concepts such as Web3 were introduced.
\\\\
Consensus-based protocols like blockchains introduced a method for each participant in a network to hold and transfer state in a digitally native format, applying stateful protocols to the traditionally stateless protocols that power the web, gives the web a native mechanism to authoritatively say who owns what, and who has permissions to do an action.
\\\\
But as groundbreaking a technology Etheruem is, it has some major problems in terms of scalability of feature-rich functionality. More generally with the current web3 paradigm, trying to build decentralised applications often forces developers to adopt multiple protocols, multiple native tokens, and multiple technology stacks in return for simplistic functionality compared to our stateless ancestor protocols.
\\\\
Our approach for building Catalyst was to solve these core issues, and to:

\begin{itemize}
\item Build a true Byzantine fault-tolerant consensus that becomes increasing scalable, decentralised and secure at scale.
\item Provide a feature-rich full-stack solution to developers who want to embrace the web3 ethos.
\item Rethink the economic incentives around blockchain systems, to be fairer to all participants.
\end{itemize}


\begin{comment}
NB - To be moved to the DFS section?

Catalyst solves this problem through the integration of a Distributed File System (DFS). This enables much more control for peers on the network in terms of what elements of the ledger they hold. This means that lightweight nodes can run with only a subset of the network data. Furthermore, large data files can be securely held on the ledger without causing issues for other nodes due to bloating.


\end{comment}