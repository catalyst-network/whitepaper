DLTs are based upon the principle of a truly distributed network, allowing nodes on the network to transact both messages and objects, whether these are tokens, messages, timestamps or even rich file types. Distributed networks have many advantages over centralised systems. They prevent any one entity from holding too much influence within a system, inhibiting the influence of an unfair or malicious actor. Further, they allow trustless transactions, meaning that no member of a network has to trust any other single member on the network, and instead they facilitate trust of the network as a whole. However, distributed networks also prevent difficult challenges: particularly how do you fairly and efficiently ensure consensus among the network is reached, how are the files stored without causing the network to become extremely bloated, and how do you compute complex business logic at least as efficiently as on a centralised network. These are all problems that have been considered in the design process of Catalyst. \\ \\

The technologies discussed in this paper are:

\begin{itemize}
\item The Probabilistic Byzantine Fault Tolerant (PBFT) consensus algorithm. This new consensus algorithm allows a fair and scalable mechanism for consensus and updates to the ledger to be agreed upon.
\item The Distributed File System (DFS) - Catalyst integrates a distributed file system meaning that there is a separation between the ledger and what is stored on the ledger.
\item The Kat Virtual Machine (KVM) - Based upon the innovative Ethereum Virtual Machine (EVM), the KVM improves and extends on the EVM. It allows smart contracts to be run on Catalyst. \\
\end{itemize}

These technologies solve three key issues identified within existing blockchain and DLT projects. \\

The first problem described was ensuring a fair consensus mechanism. The network coming to agreement on what transactions on a distributed network are fair and valid and those that are not are is a key issue. Early blockchain and DLTs utilised proof of work, this originates from hashcash \cite{back2002hashcash}, which was designed as a prevention mechanism for email spam. This idea was then developed for the bitcoin blockchain, it required miners to perform a computationally hard problem in order for them to expend energy and there by resources. This dissuades mining nodes from acting maliciously, as both the transactions they have added as well as the work they have performed can be verified. If there work is poor then the block they are trying to append to the blockchain will be rejected. This technique for consensus is Byzantine fault tolerant and can not be understated as the critical factor that has allowed blockchain technologies to grow. However, this has two critical flaws, firstly the amount of expended and wasted energy is extraordinary especially on large networks where the potential rewards are large. Secondly there has been a general trend towards centralisation at scale, as mining pools hold increase their share of the network power (and thereby the networks rewards). \\

The second issue considered while building the Catalyst network is how bloating on the network is avoided. Running a full node on many blockchains requires a significant amount of disk space, for example the Ethereum blockchain size has exceeded 1Tb \cite{EthBloat} and continues to grow. Catalyst solves this problem through the integration of a Distributed File System (DFS). This enables much more control for peers on the network what elements of the ledger that they hold. This means lightweight nodes can be ran. Furthermore, large data files can be securely held on the ledger without causing issues for other nodes due to bloating. \\

The final issue Catalyst aims to solve is the ability for complex business logic and smart contract to be ran on the ledger. While smart contracts are capable of being ran on a wide variety of blockchains frequently they are simple and incapable of running complex solutions. Often also requiring a user to write smart contracts in a specialised language. \\