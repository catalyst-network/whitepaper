DLTs are based upon the principle of a truly distributed network, allowing nodes on the network to transact both messages and objects, whether these are tokens, messages, timestamps or even rich file types. Distributed networks have many advantages over centralised systems. They prevent any one entity from holding too much influence within a system, inhibiting the influence of an unfair or malicious actor. Further, they allow trustless transactions, meaning that no member of a network has to trust any other single member on the network, and instead they facilitate trust of the network as a whole. However, distributed networks also present difficult challenges: particularly how is consensus fairly and efficiently reached among the network, how are the files stored without causing the network to become extremely bloated, and how is complex business logic computed at least as efficiently as on a centralised network. \\

The first issue discussed is the importance for how blockchain and DLTs agree on consensus across the network. Early blockchain and DLTs utilised proof of work from hashcash \cite{back2002hashcash}, which was designed as a prevention mechanism for email spam. This idea was then developed for the Bitcoin blockchain \cite{nakamoto2008bitcoin}, which required miners to perform a computationally hard problem in order for them to expend energy and  resources. This dissuades mining nodes from acting maliciously, as both the transactions they have added and the work they have performed can be verified. If their work is poor, then the block they are trying to append to the blockchain will be rejected. This technique for consensus is Byzantine fault tolerant and can not be understated as the critical factor that has allowed blockchain technologies to grow. However, this has two critical flaws. Firstly, the amount of expended and wasted energy is extraordinary, especially on large networks where the potential rewards are large. Secondly, there has been a general trend towards centralisation at scale, as mining pools increase their share of network power (and also the networks rewards). \\

The second issue deals with how bloating on the network is avoided. Running a full node on many blockchains requires a significant amount of disk space. For example, the Ethereum blockchain size has exceeded 1Tb \cite{EthBloat} and continues to grow. This problem extends to what services and data can be stored on the network. If all data is stored directly to the ledger then what information is stored is critically important. This means that files over a certain size may not be permitted or may not be easily accessible to all. \\

The final issue Catalyst aims to solve is the ability for complex business logic and smart contracts to be run on the ledger. While smart contracts are capable of being run on a wide variety of blockchains, frequently they are simple and incapable of running complex solutions. They also often require a user to write smart contracts in a specialised language. This limits the capability and range of uses for blockchains and DLTs and their applications within business. \\

Catalyst seeks to resolve each of these problems. Specifically, Catalyst has three key differences from other existing blockchain and DLT projects to solve the problems respectively:

\begin{itemize}
\item The Probabilistic Byzantine Fault Tolerant (PBFT) consensus algorithm. This new consensus algorithm allows a fair and scalable mechanism for consensus and updates to the ledger to be agreed upon.
\item The Distributed File System (DFS) - Catalyst integrates a distributed file system meaning that there is a separation between the ledger and what is stored on the ledger.
\item The Kat Virtual Machine (KVM) - Based upon the innovative Ethereum Virtual Machine (EVM), the KVM improves and extends on the EVM. It allows smart contracts to be run on Catalyst. \\
\end{itemize}

\begin{comment}
NB - To be moved to the DFS section?

Catalyst solves this problem through the integration of a Distributed File System (DFS). This enables much more control for peers on the network in terms of what elements of the ledger they hold. This means that lightweight nodes can run with only a subset of the network data. Furthermore, large data files can be securely held on the ledger without causing issues for other nodes due to bloating.


\end{comment}