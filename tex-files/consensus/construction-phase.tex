The first phase of the Catalyst consensus algorithm is the Construction Phase. Within which the selected producer nodes  $P$ calculate their proposed ledger state update or their local ledger state update. This is done by aggregating and validating all transactions that have occurred during a set time period. These transactions assuming their validity are integrated into the producers local ledger state update. From which they can create a hash of the update. This hash digest represents what they believe to be the correct update and is broadcast to the other producer nodes during for that cycle. Assuming the collision free nature of hash functions, the only mechanism for multiple producer nodes to have the same local ledger state update is by both using the same set of transactions. \\

Each producer $p$ in the set of producers $P$ follows the same protocol. The construction phase begins with producer $p$ beginning their construction phase by flushing their mempool. This mempool $T$ is made up of $n$ transactions $t$  where $n$ is the number of transactions that have been broadcast to the network and have been stored by $p$. These transaction are used to create $p$'s local ledger state update $u$. \\

The producer at this point also creates a hash tree $d$, this is to store the the signatures that are extracted from each transaction in $T$. A salt $\sigma$ is created utilizing a pseudo-random number generator using the previous ledger state update $U-1$ as its seed. $p$ then proceeds with the following steps:

\begin{enumerate}
\item Producer $p$ verifies that each transaction in $T$ is valid following the rules set out in \cite{transactionvalidator}. From each of the $n$ transactions in $T$ the entries $E$ that constitute the transaction $t$ are extracted to form a list $E = e_1,...,e_m$ for $m$ entries in $t$. The producer should therefore end up with $n$ lists for $E$ from $T$. each signature from the transactions are also extracted and added to $d$.

\item $p$ for each $E$ it created then creates a corresponding hash digest as:
\begin{center}
$H = \mathcal{H}[E~||~\sigma]$
\end{center}

Each pair ($E,H$) is added to a list $L$.

\item $p$ then sorts list $L$ into lexicographical order according the hash values $H$.

\item The producer $p$ then extracts the transaction fee value from each transaction in $T$ to create $v$ which is the total sum of all transaction fees.

\item The local ledger state update $u$ for producer $p$ can then be calculated. Firstly the list $L$ is concatenated (denoted by $||$) with the hashtree $d$ and a hash digest is created as:

\begin{center}
$u = \mathcal{H}(L~||~d)$
\end{center}

$u$ is then concatenated with $p$'s unique peer identifier $Id$ to create:

\begin{center}
$h = u ~||~Id$
\end{center}

\item $h$ is then broadcast to the other producer nodes on the network.
\end{enumerate}



Producer $p$ also collects other producers partial ledger update values. At most they will collect $P-1$ values. Optimally every producer in $P$ will receive the same set of transactions therefore for every $p$ in $P$ will have the same partial ledger update $u$. However this is unlikely due to all transactions not being received by a small group of nodes. Equally they may not hold $G$ where $G = h_1,...,h_P$, meaning they may not receive a proposed update from all candidates.
